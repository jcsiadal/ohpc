\documentclass[letterpaper]{article}
\usepackage{common/ohpc-doc}
\setcounter{secnumdepth}{5}
\setcounter{tocdepth}{5}

% Include git variables
\input{vc.tex}

% Define Base OS and other local macros
\newcommand{\baseOS}{CentOS8.4}
\newcommand{\OSRepo}{CentOS\_8.4}
\newcommand{\OSTree}{CentOS\_8}
\newcommand{\OSTag}{el8}
\newcommand{\baseos}{centos8.4}
\newcommand{\baseosshort}{centos8}
\newcommand{\provisioner}{xCAT}
\newcommand{\provheader}{xCAT (stateful)}
\newcommand{\rms}{SLURM}
\newcommand{\rmsshort}{slurm}
\newcommand{\arch}{x86\_64}
\newcommand{\installimage}{install}
%%% WARNING: Hack below. The version should be read from ohpc-doc.sty, but the
%%% perl parsing script does not read that file. This works for one release, but
%%% needs a proper fix.
\newcommand{\VERLONG}{2.0}

% Define package manager commands
\newcommand{\pkgmgr}{yum}
\newcommand{\addrepo}{wget -P /etc/yum.repos.d}
\newcommand{\chrootaddrepo}{wget -P \$CHROOT/etc/yum.repos.d}
\newcommand{\clean}{yum clean expire-cache}
\newcommand{\chrootclean}{yum --installroot=\$CHROOT clean expire-cache}
\newcommand{\install}{yum -y install}
\newcommand{\chrootinstall}{psh compute yum -y install}
\newcommand{\groupinstall}{yum -y groupinstall}
\newcommand{\groupchrootinstall}{psh compute yum -y groupinstall}
\newcommand{\remove}{yum -y remove}
\newcommand{\upgrade}{yum -y upgrade}
\newcommand{\chrootupgrade}{yum -y --installroot=\$CHROOT upgrade}
\newcommand{\tftppkg}{syslinux-tftpboot}
\newcommand{\beegfsrepo}{https://www.beegfs.io/release/beegfs\_7.2.1/dists/beegfs-rhel8.repo}

% boolean for os-specific formatting
\toggletrue{isCentOS}
\toggletrue{isCentOS_ww_slurm_x86}
\toggletrue{isSLURM}
\toggletrue{isx86}
\toggletrue{isxCAT}
\toggletrue{isxCATstateful}
\toggletrue{isCentOS_x86}

\begin{document}
\graphicspath{{common/figures/}}
\thispagestyle{empty}

% Title Page
\input{common/title}
% Disclaimer 
\input{common/legal} 

\newpage
\tableofcontents
\newpage

% Introduction  --------------------------------------------------

\section{Introduction} \label{sec:introduction}
\input{common/install_header}
\input{common/intro} \\

\input{common/base_edition/edition}
\input{common/audience}
\input{common/requirements}
\input{common/inputs}


% Base Operating System --------------------------------------------

\vspace*{0.2cm}
\section{Install Base Operating System (BOS)}
\input{common/bos}

%\clearpage 
% begin_ohpc_run
% ohpc_validation_newline
% ohpc_validation_comment Disable firewall 
\begin{lstlisting}[language=bash,keywords={}]
[sms](*\#*) systemctl disable firewalld
[sms](*\#*) systemctl stop firewalld
\end{lstlisting}
% end_ohpc_run

% ------------------------------------------------------------------

\section{Install \xCAT{} and Provision Nodes with BOS} \label{sec:provision_compute_bos}
\input{common/xcat_stateful_compute_bos_intro}

\subsection{Enable \xCAT{} repository for local use} \label{sec:enable_xcat}
\input{common/enable_xcat_repo}

\noindent \xCAT{} has a number of dependencies that are required for
installation that are housed in separate public repositories for various
distributions. To enable for local use, issue the following:

% begin_ohpc_run
\begin{lstlisting}[language=bash,keywords={},basicstyle=\fontencoding{T1}\fontsize{8.0}{10}\ttfamily,literate={ARCH}{\arch{}}1 {-}{-}1]
[sms](*\#*) (*\install*) centos-release-stream
[sms](*\#*) (*\addrepo*) https://xcat.org/files/xcat/repos/yum/xcat-dep/rh8/ARCH/xcat-dep.repo
\end{lstlisting}
% end_ohpc_run

\subsection{Add provisioning services on {\em master} node} \label{sec:add_provisioning}
\input{common/install_provisioning_xcat_intro_stateful}
%\input{common/enable_pxe}

\vspace*{-0.15cm}
\subsection{Complete basic \xCAT{} setup for {\em master} node} \label{sec:setup_xcat}
\input{common/xcat_setup}


\subsection{Define {\em compute} image for provisioning}
\input{common/xcat_init_os_images_centos}

\clearpage
\subsection{Add compute nodes into \xCAT{} database} \label{sec:xcat_add_nodes}
\input{common/add_xcat_hosts_intro}

%\vspace*{-0.25cm}
\subsection{Boot compute nodes} \label{sec:boot_computes}
\input{common/reset_computes_xcat} 




\section{Install \OHPC{} Components} \label{sec:basic_install}
\input{common/install_ohpc_components_intro}


\subsection{Enable \OHPC{} repository for local use} \label{sec:enable_repo}
\input{common/enable_local_ohpc_repo}

% begin_ohpc_run
% ohpc_validation_newline
% ohpc_validation_comment Verify OpenHPC repository has been enabled before proceeding
% ohpc_validation_newline
% ohpc_command yum repolist | grep -q OpenHPC
% ohpc_command if [ $? -ne 0 ];then
% ohpc_command    echo "Error: OpenHPC repository must be enabled locally"
% ohpc_command    exit 1
% ohpc_command fi
% end_ohpc_run


In addition to the \OHPC{} and \xCAT{} package repositories, the {\em master} host also
requires access to the standard base OS distro repositories in order to resolve
necessary dependencies. For \baseOS{}, the requirements are to have access to
both the base OS and EPEL repositories for which mirrors are freely available online:

\begin{itemize*}
\item CentOS-8 - Base 8.3.2011
  (e.g. \href{http://mirror.centos.org/centos-8/8/BaseOS/x86\_64/os}
             {\color{blue}{http://mirror.centos.org/centos-8/8/BaseOS/x86\_64/os}} )
\item EPEL 8 (e.g. \href{http://download.fedoraproject.org/pub/epel/8/Everything/x86\_64}
                        {\color{blue}{http://download.fedoraproject.org/pub/epel/8/Everything/x86\_64}} )
\end{itemize*}

\noindent The public EPEL repository is enabled by installing
\texttt{epel-release} package. Note that this requires the CentOS Extras
repository, which is shipped with CentOS and is enabled by default.

% begin_ohpc_run
\begin{lstlisting}[language=bash,keywords={},basicstyle=\fontencoding{T1}\fontsize{8.0}{10}\ttfamily,literate={ARCH}{\arch{}}1 {-}{-}1]
[sms](*\#*)  (*\install*) epel-release
\end{lstlisting}
% end_ohpc_run

Now \OHPC{} packages can be installed. To add the base package on the SMS
issue the following
% begin_ohpc_run
\begin{lstlisting}[language=bash,keywords={},basicstyle=\fontencoding{T1}\fontsize{8.0}{10}\ttfamily,literate={ARCH}{\arch{}}1 {-}{-}1]
[sms](*\#*)  (*\install*) ohpc-base
\end{lstlisting}
% end_ohpc_run


\input{common/automation}


\subsection{Setup time synchronization service on {\em master} node} \label{sec:add_ntp}
HPC systems rely on synchronized clocks throughout the system and the
NTP protocol can be used to facilitate this synchronization. To enable NTP
services on the SMS host with a specific server \texttt{\$\{ntp\_server\}}, and
allow this server to serve as a local time server for the cluster, 
issue the following:

% begin_ohpc_run
% ohpc_validation_comment Enable NTP services on SMS host
\begin{lstlisting}[language=bash,literate={-}{-}1,keywords={},upquote=true,keepspaces]
[sms](*\#*) systemctl enable chronyd.service
[sms](*\#*) echo "local stratum 10" >> /etc/chrony.conf
[sms](*\#*) echo "server ${ntp_server}" >> /etc/chrony.conf
[sms](*\#*) echo "allow all" >> /etc/chrony.conf
[sms](*\#*) systemctl restart chronyd
\end{lstlisting}
% end_ohpc_run

\begin{center}
\begin{tcolorbox}[]
\small Note that the ``allow all'' option specified for the chrony time daemon
allows all servers on the local network to be able to synchronize with the SMS
host. Alternatively, you can restrict access to fixed IP ranges and an example
config line allowing access to a local class B subnet is as follows:
\begin{lstlisting}[language=bash]
allow 192.168.0.0/16
\end{lstlisting}
\end{tcolorbox}
\end{center}


\subsection{Add resource management services on {\em master} node} \label{sec:add_rm}
\input{common/install_slurm}

\subsection{Optionally add \InfiniBand{} support services on {\em master} node} \label{sec:add_ofed}
The following command adds OFED and PSM support using base distro-provided drivers
to the chosen {\em master} host.

% begin_ohpc_run
% ohpc_comment_header Optionally add InfiniBand support services on master node \ref{sec:add_ofed}
% ohpc_command if [[ ${enable_ib} -eq 1 ]];then
% ohpc_indent 5
\begin{lstlisting}[language=bash,keywords={}]
[sms](*\#*) (*\groupinstall*) "InfiniBand Support"
\end{lstlisting}
% ohpc_indent 0
% ohpc_command fi
% end_ohpc_run

\input{common/opensm}

With the \InfiniBand{} drivers included, you can also enable (optional) IPoIB functionality
which provides a mechanism to send IP packets over the IB network. If you plan
to mount a \Lustre{} file system over \InfiniBand{} (see \S\ref{sec:lustre_client}
for additional details), then having IPoIB enabled is a requirement for the
\Lustre{} client. \OHPC{} provides a template configuration file to aid in setting up
an {\em ib0} interface on the {\em master} host. To use, copy the template
provided and update the \texttt{\$\{sms\_ipoib\}} and
\texttt{\$\{ipoib\_netmask\}} entries to match local desired settings (alter ib0
naming as appropriate if system contains dual-ported or multiple HCAs).

% begin_ohpc_run
% ohpc_validation_newline
% ohpc_validation_comment Optionally enable IPoIB interface on SMS
% ohpc_command if [[ ${enable_ipoib} -eq 1 ]];then
% ohpc_indent 5
% ohpc_validation_comment Enable ib0
\begin{lstlisting}[language=bash,literate={-}{-}1,keywords={},upquote=true]
[sms](*\#*) cp /opt/ohpc/pub/examples/network/centos/ifcfg-ib0 /etc/sysconfig/network-scripts

# Define local IPoIB address and netmask
[sms](*\#*) perl -pi -e "s/master_ipoib/${sms_ipoib}/" /etc/sysconfig/network-scripts/ifcfg-ib0
[sms](*\#*) perl -pi -e "s/ipoib_netmask/${ipoib_netmask}/" /etc/sysconfig/network-scripts/ifcfg-ib0

# configure NetworkManager to *not* override local /etc/resolv.conf
[sms](*\#*) echo "[main]"   >  /etc/NetworkManager/conf.d/90-dns-none.conf
[sms](*\#*) echo "dns=none" >> /etc/NetworkManager/conf.d/90-dns-none.conf
# Start up NetworkManager to initiate ib0
[sms](*\#*) systemctl start NetworkManager

\end{lstlisting}
% ohpc_indent 0
% ohpc_command fi
% end_ohpc_run


\vspace*{0.3cm}
\subsection{Optionally add \OmniPath{} support services on {\em master} node} \label{sec:add_opa}
The following command adds Omni-Path support using base distro-provided drivers
to the chosen {\em master} host.

% begin_ohpc_run
% ohpc_comment_header Optionally add Omni-Path support services on master node \ref{sec:add_opa}
% ohpc_command if [[ ${enable_opa} -eq 1 ]];then
% ohpc_indent 5
\begin{lstlisting}[language=bash,keywords={}]
[sms](*\#*) (*\install*) opa-basic-tools
\end{lstlisting}
% ohpc_indent 0
% ohpc_command fi
% end_ohpc_run

\input{common/opafm}



%\vspace*{0.5cm}
\clearpage
\subsubsection{Add \OHPC{} components to {\em compute} nodes} \label{sec:add_components}
\input{common/add_to_compute_stateful_xcat_intro}

%\newpage
% begin_ohpc_run
% ohpc_validation_comment Add OpenHPC components to compute instance
\begin{lstlisting}[language=bash,literate={-}{-}1,keywords={},upquote=true]
# Add Slurm client support meta-package
[sms](*\#*) (*\chrootinstall*) ohpc-slurm-client

# Add Network Time Protocol (NTP) support
[sms](*\#*) (*\chrootinstall*) ntp

# Add kernel drivers
[sms](*\#*) (*\chrootinstall*) kernel

# Include modules user environment
[sms](*\#*) (*\chrootinstall*)  --enablerepo=powertools lmod-ohpc
\end{lstlisting}
% end_ohpc_run

% ohpc_comment_header Optionally add InfiniBand support services in compute node image \ref{sec:add_components}
% ohpc_command if [[ ${enable_ib} -eq 1 ]];then
% ohpc_indent 5
\begin{lstlisting}[language=bash,literate={-}{-}1,keywords={},upquote=true]
# Optionally add IB support and enable
[sms](*\#*) (*\groupchrootinstall*) "InfiniBand Support"
\end{lstlisting}
% ohpc_indent 0
% ohpc_command fi
% end_ohpc_run

\vspace*{-0.25cm}
\subsubsection{Customize system configuration} \label{sec:master_customization}
\input{common/xcat_stateful_customize_centos}

% Additional commands when additional computes are requested

% begin_ohpc_run
% ohpc_validation_newline
% ohpc_validation_comment Update basic slurm configuration if additional computes defined
% ohpc_validation_comment This is performed on the SMS, nodes will pick it up config file is copied there later
% ohpc_command if [ ${num_computes} -gt 4 ];then
% ohpc_command    perl -pi -e "s/^NodeName=(\S+)/NodeName=${compute_prefix}[1-${num_computes}]/" /etc/slurm/slurm.conf
% ohpc_command    perl -pi -e "s/^PartitionName=normal Nodes=(\S+)/PartitionName=normal Nodes=${compute_prefix}[1-${num_computes}]/" /etc/slurm/slurm.conf
% ohpc_command fi
% end_ohpc_run

%\clearpage
\subsubsection{Additional Customization ({\em optional})} \label{sec:addl_customizations}
\input{common/compute_customizations_intro}

\paragraph{Increase locked memory limits}
\input{common/memlimits_stateful}

\paragraph{Enable ssh control via resource manager} 
\input{common/slurm_pam_stateful}

\paragraph{Add \Lustre{} client} \label{sec:lustre_client}
\input{common/lustre-client}
\input{common/lustre-client-centos-stateful}
\input{common/lustre-client-post-stateful}

\paragraph{Add \Nagios{} monitoring} \label{sec:add_nagios}
\input{common/nagios_stateful}

\vspace*{0.4cm}

\paragraph{Add \clustershell{}}
\input{common/clustershell}

\paragraph{Add \genders{}}
\input{common/genders}

\paragraph{Add Magpie}
\input{common/magpie}

\paragraph{Add \conman{}} \label{sec:add_conman}
\input{common/conman}

\paragraph{Add \nhc{}} \label{sec:add_nhc}
\input{common/nhc}
\input{common/nhc_slurm}

%\subsubsection{Identify files for synchronization} \label{sec:file_import}
%\input{common/import_xcat_files}
%\input{common/import_xcat_files_slurm}

%%%\subsubsection{Optional kernel arguments} \label{sec:optional_kargs}
%%%\input{common/conman_post}

\section{Install \OHPC{} Development Components}
\input{common/dev_intro.tex}

%\vspace*{-0.15cm}
%\clearpage
\subsection{Development Tools} \label{sec:install_dev_tools}
\input{common/dev_tools}

\vspace*{-0.15cm}
\subsection{Compilers} \label{sec:install_compilers}
\input{common/compilers}

%\clearpage
\subsection{MPI Stacks} \label{sec:mpi}
\input{common/mpi}

\subsection{Performance Tools} \label{sec:install_perf_tools}
\input{common/perf_tools}

\subsection{Setup default development environment}
\input{common/default_dev}

%\vspace*{0.2cm}
\subsection{3rd Party Libraries and Tools} \label{sec:3rdparty}
\input{common/third_party_libs_intro}

\input{common/third_party_libs}
\input{common/third_party_mpi_libs_x86}

\vspace*{.6cm}
\subsection{Optional Development Tool Builds} \label{sec:3rdparty_intel}
In addition to the 3rd party development libraries built using the open source
toolchains mentioned in \S\ref{sec:3rdparty}, \OHPC{} also provides {\em
  optional} compatible builds for use with the compilers and MPI stack included
in newer versions of the \IntelR{}~oneAPI HPC Toolkit (using the {\em classic}
compiler variants).  These
packages provide a similar hierarchical user
environment experience as other compiler and MPI families present in \OHPC{}.

To take advantage of the available builds, \OHPC{} provides a convenience
package to enable the oneAPI repository locally along with compatibility
packages that integrate oneAPI-generated compiler and MPI modulefiles within
the standard \OHPC{} user environment. To enable the \IntelR{} oneAPI
repository and install minimum compiler and MPI requirements for \OHPC{}
packaging, issue the following:

% begin_ohpc_run
% ohpc_comment_header Install Intel oneAPI tools \ref{sec:3rdparty_intel}
% ohpc_command if [[ ${enable_intel_packages} -eq 1 ]];then
% ohpc_indent 5
\begin{lstlisting}[language=bash,keywords={},upquote=true,keepspaces]
# Enable Intel oneAPI and install OpenHPC compatibility packages
[sms](*\#*) (*\install*) intel-oneapi-toolkit-release-ohpc
[sms](*\#*) rpm --import https://yum.repos.intel.com/intel-gpg-keys/GPG-PUB-KEY-INTEL-SW-PRODUCTS.PUB
[sms](*\#*) (*\install*) intel-compilers-devel-ohpc
[sms](*\#*) (*\install*) intel-mpi-devel-ohpc
\end{lstlisting}


\begin{center}
\begin{tcolorbox}[]
As noted in \S\ref{sec:master_customization}, the default installation path for
\OHPC{} (\texttt{/opt/ohpc/pub}) is exported over NFS from the {\em master} to the 
compute nodes, but the \IntelR{} oneAPI HPC Toolkit packages install to a top-level path of 
\texttt{/opt/intel}. To make the \IntelR{} compilers available to the compute 
nodes one must add an additional NFS export
for \texttt{/opt/intel} that is mounted on desired compute nodes.
\end{tcolorbox}
\end{center}

\noindent To enable all 3rd party builds available in \OHPC{} that are compatible with
the \IntelR{}~oneAPI classic compiler suite, issue the following:


% ohpc_command if [[ ${enable_opa} -eq 1 ]];then
% ohpc_indent 10
\begin{lstlisting}[language=bash,keywords={},upquote=true,keepspaces]
# Optionally, choose the Omni-Path enabled build for MVAPICH2. Otherwise, skip to retain IB variant
[sms](*\#*) (*\install*) mvapich2-psm2-intel-ohpc
\end{lstlisting}
% ohpc_indent 5
% ohpc_command fi

%\iftoggle{isSLES_ww_slurm_x86}{\clearpage}

\begin{lstlisting}[language=bash,keywords={},upquote=true,keepspaces]
# Install 3rd party libraries/tools meta-packages built with Intel toolchain
[sms](*\#*) (*\install*) ohpc-intel-serial-libs
[sms](*\#*) (*\install*) ohpc-intel-geopm
[sms](*\#*) (*\install*) ohpc-intel-io-libs
[sms](*\#*) (*\install*) ohpc-intel-perf-tools
[sms](*\#*) (*\install*) ohpc-intel-python3-libs
[sms](*\#*) (*\install*) ohpc-intel-mpich-parallel-libs
[sms](*\#*) (*\install*) ohpc-intel-mvapich2-parallel-libs
[sms](*\#*) (*\install*) ohpc-intel-openmpi4-parallel-libs
[sms](*\#*) (*\install*) ohpc-intel-impi-parallel-libs
\end{lstlisting}
% ohpc_indent 0
% ohpc_command fi
% end_ohpc_run



\section{Resource Manager Startup} \label{sec:rms_startup}
\input{common/slurm_startup_stateful}

\section{Run a Test Job} \label{sec:test_job}
With the resource manager enabled for production usage, users should now be
able to run jobs. To demonstrate this, we will add a ``test'' user on the {\em master}
host that can be used to run an example job.

% begin_ohpc_run
\begin{lstlisting}[language=bash,keywords={}]
[sms](*\#*) useradd -m test
\end{lstlisting}
% end_ohpc_run

Next, the user's credentials need to be distributed across the cluster.
\xCAT{}'s \texttt{xdcp} has a merge functionality that adds new entries into
credential files on {\em compute} nodes: 

% begin_ohpc_run
\begin{lstlisting}[language=bash,keywords={}]
# Create a sync file for pushing user credentials to the nodes
[sms](*\#*) echo "MERGE:" > syncusers
[sms](*\#*) echo "/etc/passwd -> /etc/passwd" >> syncusers
[sms](*\#*) echo "/etc/group -> /etc/group"       >> syncusers
[sms](*\#*) echo "/etc/shadow -> /etc/shadow" >> syncusers
# Use xCAT to distribute credentials to nodes
[sms](*\#*) xdcp compute -F syncusers
\end{lstlisting}
% end_ohpc_run

\nottoggle{isxCATstateful}{Alternatively, the \texttt{updatenode compute -f} command
can be used. This re-synchronizes (i.e. copies) all the files defined in the
\texttt{syncfile} setup in Section \ref{sec:file_import}. \\ }  

~\\
\input{common/prun}

\iftoggle{isSLES_ww_slurm_x86}{\clearpage}
%\iftoggle{isxCAT}{\clearpage}

\subsection{Interactive execution}
To use the newly created ``test'' account to compile and execute the
application {\em interactively} through the resource manager, execute the
following (note the use of \texttt{prun} for parallel job launch which summarizes
the underlying native job launch mechanism being used):

\begin{lstlisting}[language=bash,keywords={}]
# Switch to "test" user
[sms](*\#*) su - test

# Compile MPI "hello world" example
[test@sms ~]$ mpicc -O3 /opt/ohpc/pub/examples/mpi/hello.c

# Submit interactive job request and use prun to launch executable
[test@sms ~]$ salloc -n 8 -N 2 

[test@c1 ~]$ prun ./a.out

[prun] Master compute host = c1
[prun] Resource manager = slurm
[prun] Launch cmd = mpiexec.hydra -bootstrap slurm ./a.out

 Hello, world (8 procs total)
    --> Process #   0 of   8 is alive. -> c1
    --> Process #   4 of   8 is alive. -> c2
    --> Process #   1 of   8 is alive. -> c1
    --> Process #   5 of   8 is alive. -> c2
    --> Process #   2 of   8 is alive. -> c1
    --> Process #   6 of   8 is alive. -> c2
    --> Process #   3 of   8 is alive. -> c1
    --> Process #   7 of   8 is alive. -> c2
\end{lstlisting}

\begin{center}
\begin{tcolorbox}[]
The following table provides approximate command equivalences between SLURM and
OpenPBS:

\vspace*{0.15cm}
\input common/rms_equivalence_table
\end{tcolorbox}
\end{center}
\nottoggle{isCentOS}{\clearpage}

\iftoggle{isCentOS}{\clearpage}

\subsection{Batch execution}

For batch execution, \OHPC{} provides a simple job script for reference (also
housed in the \path{/opt/ohpc/pub/examples} directory. This example script can
be used as a starting point for submitting batch jobs to the resource manager
and the example below illustrates use of the script to submit a batch job for
execution using the same executable referenced in the previous interactive example.

\begin{lstlisting}[language=bash,keywords={}]
# Copy example job script
[test@sms ~]$ cp /opt/ohpc/pub/examples/slurm/job.mpi .

# Examine contents (and edit to set desired job sizing characteristics)
[test@sms ~]$ cat job.mpi
#!/bin/bash

#SBATCH -J test               # Job name
#SBATCH -o job.%j.out         # Name of stdout output file (%j expands to %jobId)
#SBATCH -N 2                  # Total number of nodes requested
#SBATCH -n 16                 # Total number of mpi tasks #requested
#SBATCH -t 01:30:00           # Run time (hh:mm:ss) - 1.5 hours

# Launch MPI-based executable

prun ./a.out

# Submit job for batch execution
[test@sms ~]$ sbatch job.mpi
Submitted batch job 339
\end{lstlisting}

\begin{center}
\begin{tcolorbox}[]
\small
The use of the \texttt{\%j} option in the example batch job script shown is a convenient
way to track application output on an individual job basis. The \texttt{\%j} token
is replaced with the \SLURM{} job allocation number once assigned (job~\#339 in
this example).
\end{tcolorbox}
\end{center}




\clearpage
\appendix
{\bf \LARGE \centerline{Appendices}} \vspace*{0.2cm}

\addcontentsline{toc}{section}{Appendices}
\renewcommand{\thesubsection}{\Alph{subsection}}

\input{common/automation_appendix}
\input{common/upgrade_stateful}
\clearpage
\subsection{Integration Test Suite}  \label{appendix:test_suite}

This appendix details the installation and basic use of the integration test
suite used to support \OHPC{} releases. This suite is not intended to replace
the validation performed by component development teams, but is instead,
devised to confirm component builds are functional and interoperable within
the modular \OHPC{} environment.
The test suite is generally organized by components and the \OHPC{} CI workflow
relies on running the full suite using \href{https://jenkins.io}{\color{blue}{Jenkins}} to test
multiple OS configurations and installation recipes.
%Each \OHPC{} component is equipped with a set of scripts and applications
%to test the integration of these components in a Jenkins CI 
%environment.
To facilitate customization and running of the test suite locally, we 
provide these tests in a standalone RPM. 

\begin{lstlisting}
[sms](*\#*) (*\install*) test-suite-ohpc
\end{lstlisting}

The RPM installation creates a user named \texttt{ohpc-test} to house the test
suite and provide an isolated environment for execution.  Configuration of the
test suite is done using standard \GNU{} autotools semantics and the
\href{https://jenkins.io}{\color{blue}{BATS}} shell-testing framework is used
to execute and log a number of individual unit tests.  Some tests require
privileged execution, so a different combination of tests will be enabled
depending on which user executes the top-level \texttt{configure}
script. Non-privileged tests requiring execution on one or more compute nodes are
submitted as jobs through the \rms{} resource manager. The tests are further
divided into ``short'' and ``long'' run categories. The short run configuration
is a subset
of approximately 180 tests to demonstrate basic functionality of key components
(e.g. MPI stacks) and should complete in 10-20 minutes. The long run (around
1000 tests) is comprehensive and can take an hour or more to complete.

Most components can be tested individually, but a default configuration is
setup to enable collective testing. To test an isolated component, use the
\texttt{configure} option to disable all tests, then re-enable the desired test
to run. The \texttt{--help} option to \texttt{configure} will display all
possible tests. By default, the test suite will endeavor to run tests for
multiple MPI stacks where applicable. To restrict tests to only a subset of MPI
families, use the \texttt{--with-mpi-families} option
(e.g. \texttt{--with-mpi-families="openmpi4"}). Example output is shown below
(some output is omitted for the sake of brevity).

\begin{lstlisting}[literate={RMS}{\rms{}}1 {ARCH}{\arch{}}1]
[sms](*\#*) su - ohpc-test
[test@sms ~]$ cd tests
[test@sms ~]$ ./configure --disable-all --enable-fftw
checking for a BSD-compatible install... /bin/install -c
checking whether build environment is sane... yes
...
---------------------------------------------- SUMMARY ---------------------------------------------

Package version............... : test-suite-2.0.0

Build user.................... : ohpc-test
Build host.................... : sms001
Configure date................ : 2020-10-05 08:22
Build architecture............ : ARCH
Compiler Families............. : gnu9
MPI Families.................. : mpich mvapich2 openmpi4
Python Families............... : python3
Resource manager ............. : RMS
Test suite configuration...... : short
...
Libraries:
    Adios .................... : disabled
    Boost .................... : disabled
    Boost MPI................. : disabled
    FFTW...................... : enabled
    GSL....................... : disabled
    HDF5...................... : disabled
    HYPRE..................... : disabled
...
\end{lstlisting}

\iftoggle{isCentOS_ww_pbs_aarch}{\clearpage}

Many \OHPC{} components exist in multiple flavors to support multiple compiler
and MPI runtime permutations, and the test suite takes this in to account by
iterating through these combinations by default. If \texttt{make check} is
executed from the top-level test directory, all configured compiler and MPI
permutations of a library will be exercised. The following highlights the
execution of the FFTW related tests that were enabled in the previous step.

\begin{lstlisting}[literate={RMS}{\rms{}}1]
[test@sms ~]$ make check
make --no-print-directory check-TESTS
PASS: libs/fftw/ohpc-tests/test_mpi_families
============================================================================
Testsuite summary for test-suite 2.0.0
============================================================================
# TOTAL: 1
# PASS:  1
# SKIP:  0
# XFAIL: 0
# FAIL:  0
# XPASS: 0
# ERROR: 0
============================================================================
[test@sms ~]$ cat libs/fftw/tests/family-gnu*/rm_execution.log 
1..3
ok 1 [libs/FFTW] Serial C binary runs under resource manager (RMS/gnu9/mpich)
ok 2 [libs/FFTW] MPI C binary runs under resource manager (RMS/gnu9/mpich)
ok 3 [libs/FFTW] Serial Fortran binary runs under resource manager (RMS/gnu9/mpich)
PASS rm_execution (exit status: 0)
1..3
ok 1 [libs/FFTW] Serial C binary runs under resource manager (RMS/gnu9/mvapich2)
ok 2 [libs/FFTW] MPI C binary runs under resource manager (RMS/gnu9/mvapich2)
ok 3 [libs/FFTW] Serial Fortran binary runs under resource manager (RMS/gnu9/mvapich2)
PASS rm_execution (exit status: 0)
1..3
ok 1 [libs/FFTW] Serial C binary runs under resource manager (RMS/gnu9/openmpi4)
ok 2 [libs/FFTW] MPI C binary runs under resource manager (RMS/gnu9/openmpi4)
ok 3 [libs/FFTW] Serial Fortran binary runs under resource manager (RMS/gnu9/openmpi4)
PASS rm_execution (exit status: 0)
\end{lstlisting}

\input{common/customization_appendix_centos}
../../warewulf/slurm/manifest.tex
\input{common/signature}


\end{document}

