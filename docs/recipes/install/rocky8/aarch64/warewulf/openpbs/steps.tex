\documentclass[letterpaper]{article}
\usepackage{common/ohpc-doc}
\setcounter{secnumdepth}{5}
\setcounter{tocdepth}{5}

% Include git variables
\input{vc.tex}

% Define Base OS and other local macros
\newcommand{\baseOS}{Rocky 8.5}
\newcommand{\OSRepo}{Rocky\_8.5}
\newcommand{\OSTree}{EL\_8}
\newcommand{\OSTag}{el8}
\newcommand{\baseos}{rocky8.5}
\newcommand{\baseosshort}{rocky8}
\newcommand{\provisioner}{Warewulf}
\newcommand{\provheader}{\provisioner{}}
\newcommand{\rms}{OpenPBS}
\newcommand{\rmsshort}{OpenPBS}
\newcommand{\arch}{aarch64}

% Define package manager commands
\newcommand{\pkgmgr}{yum}
\newcommand{\addrepo}{wget -P /etc/yum.repos.d}
\newcommand{\chrootaddrepo}{wget -P \$CHROOT/etc/yum.repos.d}
\newcommand{\clean}{yum clean expire-cache}
\newcommand{\chrootclean}{yum --installroot=\$CHROOT clean expire-cache}
\newcommand{\install}{yum -y install}
\newcommand{\chrootinstall}{yum -y --installroot=\$CHROOT install}
\newcommand{\groupinstall}{yum -y groupinstall}
\newcommand{\groupchrootinstall}{yum -y --installroot=\$CHROOT groupinstall}
\newcommand{\remove}{yum -y remove}
\newcommand{\upgrade}{yum -y upgrade}
\newcommand{\chrootupgrade}{yum -y --installroot=\$CHROOT upgrade}
\newcommand{\tftppkg}{syslinux-tftpboot}

% boolean for os-specific formatting
\toggletrue{isaarch}
\toggletrue{isCentOS}
\toggletrue{isCentOS_ww_pbs_aarch}
\toggletrue{ispbs}
\toggletrue{isWarewulf}

\begin{document}
\graphicspath{{common/figures/}}
\thispagestyle{empty}

% Title Page --------------------------------------------------------
\input{common/title}
% Disclaimer
\input{common/legal} 

\newpage
\tableofcontents
\newpage

% Introduction  ----------------------------------------------------


\section{Introduction} \label{sec:introduction}
\input{common/install_header}
\input{common/intro} \\

\input{common/base_edition/edition}
\input{common/audience}
\input{common/requirements}
\input{common/inputs}

% begin_ohpc_run
% ohpc_validation_newline
% ohpc_validation_comment Verify OpenHPC repository has been enabled before proceeding
% ohpc_validation_newline
% ohpc_command yum repolist | grep -q OpenHPC
% ohpc_command if [ $? -ne 0 ];then
% ohpc_command    echo "Error: OpenHPC repository must be enabled locally"
% ohpc_command    exit 1
% ohpc_command fi
% end_ohpc_run

% Base Operating System --------------------------------------------

\section{Install Base Operating System (BOS)}
\input{common/bos}

% begin_ohpc_run
% ohpc_validation_newline
% ohpc_validation_comment Disable firewall 
\begin{lstlisting}[language=bash,keywords={}]
[sms](*\#*) systemctl disable firewalld
[sms](*\#*) systemctl stop firewalld
\end{lstlisting}
% end_ohpc_run

% ------------------------------------------------------------------

\section{Install \OHPC{} Components} \label{sec:basic_install}
\input{common/install_ohpc_components_intro.tex}

\subsection{Enable \OHPC{} repository for local use} \label{sec:enable_repo}
\input{common/enable_ohpc_repo}
\input{common/rocky_repos}
\input{common/automation}

\subsection{Add provisioning services on {\em master} node} \label{sec:add_provisioning}
\input{common/install_provisioning_intro}
\input{common/enable_pxe}
HPC systems rely on synchronized clocks throughout the system and the
NTP protocol can be used to facilitate this synchronization. To enable NTP
services on the SMS host with a specific server \texttt{\$\{ntp\_server\}}, and
allow this server to serve as a local time server for the cluster, 
issue the following:

% begin_ohpc_run
% ohpc_validation_comment Enable NTP services on SMS host
\begin{lstlisting}[language=bash,literate={-}{-}1,keywords={},upquote=true,keepspaces]
[sms](*\#*) systemctl enable chronyd.service
[sms](*\#*) echo "local stratum 10" >> /etc/chrony.conf
[sms](*\#*) echo "server ${ntp_server}" >> /etc/chrony.conf
[sms](*\#*) echo "allow all" >> /etc/chrony.conf
[sms](*\#*) systemctl restart chronyd
\end{lstlisting}
% end_ohpc_run

\begin{center}
\begin{tcolorbox}[]
\small Note that the ``allow all'' option specified for the chrony time daemon
allows all servers on the local network to be able to synchronize with the SMS
host. Alternatively, you can restrict access to fixed IP ranges and an example
config line allowing access to a local class B subnet is as follows:
\begin{lstlisting}[language=bash]
allow 192.168.0.0/16
\end{lstlisting}
\end{tcolorbox}
\end{center}


\subsection{Add resource management services on {\em master} node} \label{sec:add_rm}
\input{common/install_openpbs}

%% Add if/when IB is available for testing
%% \subsection{Optionally add \InfiniBand{} support services on {\em master} node} \label{sec:add_ofed}
%% The following command adds OFED and PSM support using base distro-provided drivers
to the chosen {\em master} host.

% begin_ohpc_run
% ohpc_comment_header Optionally add InfiniBand support services on master node \ref{sec:add_ofed}
% ohpc_command if [[ ${enable_ib} -eq 1 ]];then
% ohpc_indent 5
\begin{lstlisting}[language=bash,keywords={}]
[sms](*\#*) (*\groupinstall*) "InfiniBand Support"
\end{lstlisting}
% ohpc_indent 0
% ohpc_command fi
% end_ohpc_run

\input{common/opensm}

With the \InfiniBand{} drivers included, you can also enable (optional) IPoIB functionality
which provides a mechanism to send IP packets over the IB network. If you plan
to mount a \Lustre{} file system over \InfiniBand{} (see \S\ref{sec:lustre_client}
for additional details), then having IPoIB enabled is a requirement for the
\Lustre{} client. \OHPC{} provides a template configuration file to aid in setting up
an {\em ib0} interface on the {\em master} host. To use, copy the template
provided and update the \texttt{\$\{sms\_ipoib\}} and
\texttt{\$\{ipoib\_netmask\}} entries to match local desired settings (alter ib0
naming as appropriate if system contains dual-ported or multiple HCAs).

% begin_ohpc_run
% ohpc_validation_newline
% ohpc_validation_comment Optionally enable IPoIB interface on SMS
% ohpc_command if [[ ${enable_ipoib} -eq 1 ]];then
% ohpc_indent 5
% ohpc_validation_comment Enable ib0
\begin{lstlisting}[language=bash,literate={-}{-}1,keywords={},upquote=true]
[sms](*\#*) cp /opt/ohpc/pub/examples/network/centos/ifcfg-ib0 /etc/sysconfig/network-scripts

# Define local IPoIB address and netmask
[sms](*\#*) perl -pi -e "s/master_ipoib/${sms_ipoib}/" /etc/sysconfig/network-scripts/ifcfg-ib0
[sms](*\#*) perl -pi -e "s/ipoib_netmask/${ipoib_netmask}/" /etc/sysconfig/network-scripts/ifcfg-ib0

# configure NetworkManager to *not* override local /etc/resolv.conf
[sms](*\#*) echo "[main]"   >  /etc/NetworkManager/conf.d/90-dns-none.conf
[sms](*\#*) echo "dns=none" >> /etc/NetworkManager/conf.d/90-dns-none.conf
# Start up NetworkManager to initiate ib0
[sms](*\#*) systemctl start NetworkManager

\end{lstlisting}
% ohpc_indent 0
% ohpc_command fi
% end_ohpc_run



%\vspace*{-0.15cm}
\subsection{Complete basic Warewulf setup for {\em master} node} \label{sec:setup_ww}
\input{common/warewulf_setup}
\input{common/warewulf_setup_centos}

\subsection{Define {\em compute} image for provisioning}
\input{common/warewulf_mkchroot_rocky}

\subsubsection{Add \OHPC{} components} \label{sec:add_components}
\input{common/add_to_compute_chroot_intro}

% begin_ohpc_run
% ohpc_validation_comment Add OpenHPC components to compute instance
\begin{lstlisting}[language=bash,literate={-}{-}1,keywords={},upquote=true]
# Add OpenPBS client support
[sms](*\#*) (*\chrootinstall*) openpbs-execution-ohpc
[sms](*\#*) perl -pi -e "s/PBS_SERVER=\S+/PBS_SERVER=${sms_name}/" $CHROOT/etc/pbs.conf
[sms](*\#*) echo "PBS_LEAF_NAME=${sms_name}" >> /etc/pbs.conf
[sms](*\#*) chroot $CHROOT opt/pbs/libexec/pbs_habitat
[sms](*\#*) perl -pi -e "s/\$clienthost \S+/\$clienthost ${sms_name}/" $CHROOT/var/spool/pbs/mom_priv/config
[sms](*\#*) echo "\$usecp *:/home /home" >> $CHROOT/var/spool/pbs/mom_priv/config
[sms](*\#*) chroot $CHROOT systemctl enable pbs

# Add Network Time Protocol (NTP) support
[sms](*\#*) (*\chrootinstall*) chrony
[sms](*\#*) echo "server ${sms_ip} iburst" >> $CHROOT/etc/chrony.conf

# Add kernel drivers (matching kernel version on SMS node)
[sms](*\#*) (*\chrootinstall*) kernel-`uname -r`

# Include modules user environment
[sms](*\#*) (*\chrootinstall*) lmod-ohpc
\end{lstlisting}
% end_ohpc_run

\subsubsection{Customize system configuration} \label{sec:master_customization}
Prior to assembling the image, it is advantageous to perform any additional
customization within the chroot environment created for the desired {\em
 compute} instance. The following steps document the process to add a local
{\em ssh} key created by \Warewulf{} to support remote access, 
%identify the resource manager server, configure NTP for compute resources, 
and enable \NFS{}
mounting of a \$HOME file system and the public \OHPC{} install path
(\texttt{/opt/ohpc/pub}) that will be hosted by the {\em master} host in this
example configuration.

\iftoggleverb{isCentOS_ww_pbs_x86}
\vspace*{0.15cm}
%\clearpage
\else
\vspace*{0.15cm}
\fi

% begin_ohpc_run
% ohpc_comment_header Customize system configuration \ref{sec:master_customization}
\begin{lstlisting}[language=bash,literate={-}{-}1,keywords={},upquote=true]
# Initialize warewulf database and ssh_keys
[sms](*\#*) wwinit database
[sms](*\#*) wwinit ssh_keys

# Add NFS client mounts of /home and /opt/ohpc/pub to base image
[sms](*\#*) echo "${sms_ip}:/home /home nfs nfsvers=3,nodev,nosuid 0 0" >> $CHROOT/etc/fstab
[sms](*\#*) echo "${sms_ip}:/opt/ohpc/pub /opt/ohpc/pub nfs nfsvers=3,nodev 0 0" >> $CHROOT/etc/fstab

# Export /home and OpenHPC public packages from master server
[sms](*\#*) echo "/home *(rw,no_subtree_check,fsid=10,no_root_squash)" >> /etc/exports
[sms](*\#*) echo "/opt/ohpc/pub *(ro,no_subtree_check,fsid=11)" >> /etc/exports
\end{lstlisting}
% end_ohpc_run


% begin_ohpc_run
\begin{lstlisting}[language=bash,literate={-}{-}1,keywords={},upquote=true]
# Finalize NFS config and restart
[sms](*\#*) exportfs -a
[sms](*\#*) systemctl restart nfs-server
[sms](*\#*) systemctl enable nfs-server
\end{lstlisting}
% end_ohpc_run


%\clearpage
\subsubsection{Additional Customization ({\em optional})} \label{sec:addl_customizations}
\input{common/compute_customizations_intro}

%% Add if/when IB is available for testing
%% \paragraph{Increase locked memory limits}
%% \input{common/memlimits}

%%\paragraph{Add \Lustre{} client} \label{sec:lustre_client}
%%\input{common/lustre-client}
%%\input{common/lustre-client-centos}
%%\vspace*{0.5cm}
%%\input{common/lustre-client-post}

%\clearpage
\vspace*{-.2cm}
\paragraph{Enable forwarding of system logs} \label{sec:add_syslog}
\input{common/syslog}

\vspace*{-.2cm}
\paragraph{Add \Nagios{} monitoring} \label{sec:add_nagios}
\input{common/nagios}

%\clearpage
\paragraph{Add \clustershell{}}
\input{common/clustershell}

%\clearpage
%\paragraph{Add \mrsh{}}
%\input{common/mrsh}

\paragraph{Add \genders{}}
\input{common/genders}

\vspace*{-.1cm}
\paragraph{Add Magpie}
\input{common/magpie}

\vspace*{-.1cm}
\paragraph{Add \conman{}} \label{sec:add_conman}
\input{common/conman}

\vspace*{-.1cm}
\paragraph{Add \nhc{}} \label{sec:add_nhc}
\input{common/nhc}

\subsubsection{Import files} \label{sec:file_import}
\input{common/import_ww_files}
%% \input{common/import_ww_files_ib_centos}
\input{common/finalize_provisioning}
\vspace*{0.2cm}
\input{common/add_ww_hosts_intro}
\input{common/add_ww_hosts_pbs}
\input{common/add_ww_hosts_finalize}

\vspace*{-0.2cm}
\subsubsection{Optional kernel arguments} \label{sec:optional_kargs}
\input{common/conman_post}
\input{common/kargs_post}

\vspace*{-0.2cm}
\subsubsection{Optionally configure stateful provisioning}
\input{common/stateful}

\vspace*{-0.1cm}
\subsection{Boot compute nodes} \label{sec:boot_computes}
\input{common/reset_computes} 

\vspace*{-0.50cm}
\section{Install \OHPC{} Development Components} \label{sec:install_dev}
\input{common/dev_intro.tex}

\vspace*{-0.15cm}
\subsection{Development Tools} \label{sec:install_dev_tools}
\input{common/dev_tools}

\vspace*{-0.15cm}
\subsection{Compilers} \label{sec:install_compilers}
\input{common/compilers}

%\clearpage
\subsection{MPI Stacks} \label{sec:mpi}
\input{common/mpi_aarch}

\subsection{Performance Tools} \label{sec:install_perf_tools}
\input{common/perf_tools}

\subsection{Setup default development environment}
\input{common/default_dev}

%\clearpage
\subsection{3rd Party Libraries and Tools} \label{sec:3rdparty}
\input{common/third_party_libs_intro}
\input{common/third_party_libs_petsc_centos}
\vspace*{.4cm}
\input{common/third_party_libs}
\input{common/third_party_mpi_libs_aarch}

\subsection{Optional Development Tool Builds} \label{sec:3rdparty_arm}
\input{common/armhpc_enabled_builds}

\clearpage
\section{Resource Manager Startup} \label{sec:rms_startup}
\input{common/openpbs_startup}

\section{Run a Test Job} \label{sec:test_job}
\input{common/openpbs_test_job}

\clearpage
\appendix
{\bf \LARGE \centerline{Appendices}} \vspace*{0.2cm}

\addcontentsline{toc}{section}{Appendices}
\renewcommand{\thesubsection}{\Alph{subsection}}

\input{common/automation_appendix}
\input{common/upgrade}
\clearpage
\subsection{Integration Test Suite}  \label{appendix:test_suite}

This appendix details the installation and basic use of the integration test
suite used to support \OHPC{} releases. This suite is not intended to replace
the validation performed by component development teams, but is instead,
devised to confirm component builds are functional and interoperable within
the modular \OHPC{} environment.
The test suite is generally organized by components and the \OHPC{} CI workflow
relies on running the full suite using \href{https://jenkins.io}{\color{blue}{Jenkins}} to test
multiple OS configurations and installation recipes.
%Each \OHPC{} component is equipped with a set of scripts and applications
%to test the integration of these components in a Jenkins CI 
%environment.
To facilitate customization and running of the test suite locally, we 
provide these tests in a standalone RPM. 

\begin{lstlisting}
[sms](*\#*) (*\install*) test-suite-ohpc
\end{lstlisting}

The RPM installation creates a user named \texttt{ohpc-test} to house the test
suite and provide an isolated environment for execution.  Configuration of the
test suite is done using standard \GNU{} autotools semantics and the
\href{https://jenkins.io}{\color{blue}{BATS}} shell-testing framework is used
to execute and log a number of individual unit tests.  Some tests require
privileged execution, so a different combination of tests will be enabled
depending on which user executes the top-level \texttt{configure}
script. Non-privileged tests requiring execution on one or more compute nodes are
submitted as jobs through the \rms{} resource manager. The tests are further
divided into ``short'' and ``long'' run categories. The short run configuration
is a subset
of approximately 180 tests to demonstrate basic functionality of key components
(e.g. MPI stacks) and should complete in 10-20 minutes. The long run (around
1000 tests) is comprehensive and can take an hour or more to complete.

Most components can be tested individually, but a default configuration is
setup to enable collective testing. To test an isolated component, use the
\texttt{configure} option to disable all tests, then re-enable the desired test
to run. The \texttt{--help} option to \texttt{configure} will display all
possible tests. By default, the test suite will endeavor to run tests for
multiple MPI stacks where applicable. To restrict tests to only a subset of MPI
families, use the \texttt{--with-mpi-families} option
(e.g. \texttt{--with-mpi-families="openmpi4"}). Example output is shown below
(some output is omitted for the sake of brevity).

\begin{lstlisting}[literate={RMS}{\rms{}}1 {ARCH}{\arch{}}1]
[sms](*\#*) su - ohpc-test
[test@sms ~]$ cd tests
[test@sms ~]$ ./configure --disable-all --enable-fftw
checking for a BSD-compatible install... /bin/install -c
checking whether build environment is sane... yes
...
---------------------------------------------- SUMMARY ---------------------------------------------

Package version............... : test-suite-2.0.0

Build user.................... : ohpc-test
Build host.................... : sms001
Configure date................ : 2020-10-05 08:22
Build architecture............ : ARCH
Compiler Families............. : gnu9
MPI Families.................. : mpich mvapich2 openmpi4
Python Families............... : python3
Resource manager ............. : RMS
Test suite configuration...... : short
...
Libraries:
    Adios .................... : disabled
    Boost .................... : disabled
    Boost MPI................. : disabled
    FFTW...................... : enabled
    GSL....................... : disabled
    HDF5...................... : disabled
    HYPRE..................... : disabled
...
\end{lstlisting}

\iftoggle{isCentOS_ww_pbs_aarch}{\clearpage}

Many \OHPC{} components exist in multiple flavors to support multiple compiler
and MPI runtime permutations, and the test suite takes this in to account by
iterating through these combinations by default. If \texttt{make check} is
executed from the top-level test directory, all configured compiler and MPI
permutations of a library will be exercised. The following highlights the
execution of the FFTW related tests that were enabled in the previous step.

\begin{lstlisting}[literate={RMS}{\rms{}}1]
[test@sms ~]$ make check
make --no-print-directory check-TESTS
PASS: libs/fftw/ohpc-tests/test_mpi_families
============================================================================
Testsuite summary for test-suite 2.0.0
============================================================================
# TOTAL: 1
# PASS:  1
# SKIP:  0
# XFAIL: 0
# FAIL:  0
# XPASS: 0
# ERROR: 0
============================================================================
[test@sms ~]$ cat libs/fftw/tests/family-gnu*/rm_execution.log 
1..3
ok 1 [libs/FFTW] Serial C binary runs under resource manager (RMS/gnu9/mpich)
ok 2 [libs/FFTW] MPI C binary runs under resource manager (RMS/gnu9/mpich)
ok 3 [libs/FFTW] Serial Fortran binary runs under resource manager (RMS/gnu9/mpich)
PASS rm_execution (exit status: 0)
1..3
ok 1 [libs/FFTW] Serial C binary runs under resource manager (RMS/gnu9/mvapich2)
ok 2 [libs/FFTW] MPI C binary runs under resource manager (RMS/gnu9/mvapich2)
ok 3 [libs/FFTW] Serial Fortran binary runs under resource manager (RMS/gnu9/mvapich2)
PASS rm_execution (exit status: 0)
1..3
ok 1 [libs/FFTW] Serial C binary runs under resource manager (RMS/gnu9/openmpi4)
ok 2 [libs/FFTW] MPI C binary runs under resource manager (RMS/gnu9/openmpi4)
ok 3 [libs/FFTW] Serial Fortran binary runs under resource manager (RMS/gnu9/openmpi4)
PASS rm_execution (exit status: 0)
\end{lstlisting}

\input{common/customization_appendix_centos}
../../warewulf/slurm/manifest.tex
\input{common/signature}


\end{document}

