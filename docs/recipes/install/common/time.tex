HPC systems rely on synchronized clocks throughout the system and the
NTP protocol can be used to facilitate this synchronization. To enable NTP
services on the SMS host with a specific server \texttt{\$\{ntp\_server\}}, and
allow this server to serve as a local time server for the cluster, 
issue the following:

% begin_ohpc_run
% ohpc_validation_comment Enable NTP services on SMS host
\begin{lstlisting}[language=bash,literate={-}{-}1,keywords={},upquote=true,keepspaces]
[sms](*\#*) systemctl enable chronyd.service
[sms](*\#*) echo "local stratum 10" >> /etc/chrony.conf
[sms](*\#*) echo "server ${ntp_server}" >> /etc/chrony.conf
[sms](*\#*) echo "allow all" >> /etc/chrony.conf
[sms](*\#*) systemctl restart chronyd
\end{lstlisting}
% end_ohpc_run

\begin{center}
\begin{tcolorbox}[]
\small Note that the ``allow all'' option specified for the chrony time daemon
allows all servers on the local network to be able to synchronize with the SMS
host. Alternatively, you can restrict access to fixed IP ranges and an example
config line allowing access to a local class B subnet is as follows:
\begin{lstlisting}[language=bash]
allow 192.168.0.0/16
\end{lstlisting}
\end{tcolorbox}
\end{center}
