The Charliecloud container runtime requires enabling the user namespaces mapping
option. This allows applications to run with root privilege inside a container, 
but have them run as a different, typically non-privileged, user on the host.
Though generally regarded as mature and safe, RHEL considers this a Technology
Preview, so it must be manually enabled with kernel arguments. We demonstrate
enabling on compute nodes, but it would also be required on the SMS if you wish
to build and run containers there.

% begin_ohpc_run
% ohpc_validation_newline
% ohpc_validation_comment Optionally, enable user namespaces
\begin{lstlisting}[language=bash,keywords={},upquote=true]
# Define node kernel arguments to support user namespaces
[sms](*\#*) export kargs="${kargs} namespace.unpriv_enable=1"

# Increase per-user limit on the number of user namespaces that may be created
[sms](*\#*) echo "user.max_user_namespaces=15076" >> $CHROOT/etc/sysctl.conf
# rebuild VNFS
[sms](*\#*) wwvnfs --chroot $CHROOT
\end{lstlisting}
% end_ohpc_run

\begin{center}
\begin{tcolorbox}[]
\small
Typical Charliecloud workflows are based around Docker containers, but it is not
strictly necessary to install Docker itself on the HPC resource. A common
pattern is to build the Docker container on a laptop or VM and upload the result
to the cluster for use with Charliecloud. More information can be found at
\href{https://hpc.github.io/charliecloud/}
    {\color{blue}{https://hpc.github.io/charliecloud/}}
\end{tcolorbox}
\end{center}
